\documentclass{article}
\usepackage[utf8]{inputenc}
\usepackage{amsmath, amssymb, graphicx}
\usepackage{booktabs}

\title{Sample \LaTeX{} Document}
\author{motoshin}
\date{\today}

\begin{document}

\maketitle

\section*{Abstract}
This is a sample document created using \LaTeX{}. It includes sections, equations, tables, and figures to demonstrate its capabilities.

\section{Introduction}
\LaTeX{} is a powerful tool for creating documents with high typographic quality. This document demonstrates various features, including mathematical expressions, tables, and figures.

\section{Mathematics}
\subsection{Inline Math}
An inline math example: $E = mc^2$.

\subsection{Displayed Equations}
Here is a displayed equation:
\begin{equation}
\int_a^b x^2 \, dx = \frac{b^3}{3} - \frac{a^3}{3}.
\end{equation}

A system of equations:
\begin{align}
x + y &= 10, \\
x - y &= 4.
\end{align}

\section{Tables}
Here is a sample table:
\begin{table}[h!]
\centering
\begin{tabular}{llr}
\toprule
\textbf{Item} & \textbf{Description} & \textbf{Price} \\
\midrule
Item 1 & Description 1 & \$10.00 \\
Item 2 & Description 2 & \$15.00 \\
Item 3 & Description 3 & \$20.00 \\
\bottomrule
\end{tabular}
\caption{Sample table with items and prices.}
\label{tab:sample_table}
\end{table}

\section{Figures}
Here is a sample figure:
\begin{figure}[h!]
\centering
\includegraphics[width=0.5\textwidth]{example-image}
\caption{A sample figure. Replace \texttt{example-image} with your image file.}
\label{fig:sample_figure}
\end{figure}

\section{Gorillas}
Gorillas are large, ground-dwelling primates that are native to the forests of central Sub-Saharan Africa. They are herbivorous and primarily feed on leaves, stems, fruit, and bamboo shoots. Gorillas are highly social animals, typically living in groups led by a dominant male known as a silverback.

There are two main species of gorillas:
\begin{itemize}
    \item \textbf{Eastern Gorilla} (\textit{Gorilla beringei})
    \item \textbf{Western Gorilla} (\textit{Gorilla gorilla})
\end{itemize}

Gorillas are known for their intelligence, with behaviors such as tool use and complex communication observed in the wild. Conservation efforts are critical for their survival due to threats like habitat destruction and poaching.

\section{Conclusion}
This document showcases the basic features of \LaTeX{} for creating well-structured and visually appealing documents. Experiment with adding more sections, equations, and graphics to suit your needs.

\end{document}
